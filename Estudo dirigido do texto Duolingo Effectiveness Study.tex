\documentclass[a4paper, 12pt]{article}
\usepackage[utf8]{inputenc}
\usepackage[T1]{fontenc}
\usepackage[portuguese]{babel}
\usepackage{lmodern}
\usepackage{amsmath}
\usepackage{pifont}
\title{Estudo da efetividade do Duolingo}
\author{Deise}
\date{\today}
\begin{document}
\maketitle
\begin{abstract}
    Este documento é um estudo dirigido sobre o texto "Duolingo Effectiveness Study".
\end{abstract}
\section{Introdução}
O estudo da efetividade do Duolingo foi realizado por Vesselinov e Grego (2012) com o objetivo de avaliar a eficácia do Duolingo no aprendizado de idiomas. A pesquisa foi financiada pelo Duolingo e contou com a participação de 196 pessoas que estavam aprendendo espanhol.
\section{Metodologia}
Os participantes do estudo foram selecionados com base nos seguintes critérios: 
\begin{itemize}

    \ding{51} Ter pelo menos 18 anos de idade; \\
    \ding{51} Ser nativo de língua inglesa; \\
    \ding{51} Não ser de origem hispânica; \\
    \ding{51} Não ser de estudante avançado de espanhol; \\
    \ding{51} Residir nos Estados Unidos. 
    
\end{itemize}
O estudo durou aproximadamente oito semanas e os participantes foram instruídos a usar o Duolingo por pelo menos 2 horas por semana.
A efetividade do Duolingo foi medida como a melhora da linguagem por 1 hora de estudo.
\section{Resultados}
Os resultados do estudo mostraram que o Duolingo é uma ferramenta eficaz para o aprendizado de idiomas. Em média, os participantes ganharam 8,1 pontos WebCAPE por hora de estudo com o Duolingo. Os principais fatores que influenciaram a efetividade do Duolingo foram: \\

\begin{itemize}
    \ding{51} Motivação: As pessoas que estudavam para viajar obtiveram os melhores resultados do que as pessoas que estudavam por interesse pessoal; \\
    \\
    \ding{51} Nível inicial de conhecimento: os iniciantes obtiveram melhores resultados do que os alunos mais avançados. 
\end{itemize}
\section{Conclusão}
O estudo de Vesselinov e Grego (2012) fornece evidências de que o Duolingo é uma ferramenta eficaz para o aprendizado de idiomas. \\
O estudo também sugere que a motivação e o nível inicial de conhecimento dos alunos são fatores importantes que influenciam a efetividade do Duolingo.

\section{Referência}
VESSELINOV, Roumem & GREGO, John. Duolingo effectiveness study. City University of New York, USA, v. 28, n. 1-25, 2012.
\end{document}